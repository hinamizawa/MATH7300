\documentclass[12pt,reqno]{amsart}
\usepackage{fullpage}
\usepackage{amsfonts}
\usepackage{tikz}
\usepackage{amssymb}
\usepackage{times}
\usepackage{graphicx}
\usepackage{mathtools}
\usepackage{breakurl}
\usepackage{bm}
\usepackage{amsmath}
\usepackage{blkarray}
\usepackage{url}
\usepackage[all]{xy}
\usepackage[margin=0.8in,footskip=0.25in]{geometry}

\usepackage[colorlinks=true,
            linkcolor=red,
            urlcolor=blue,
            citecolor=red]{hyperref}
\vfuzz=2pt


\DeclarePairedDelimiter\ceil{\lceil}{\rceil}
\DeclarePairedDelimiter\floor{\lfloor}{\rfloor}

\DeclareMathOperator{\cok}{coker}
\DeclareMathOperator{\im}{im}
\DeclareMathOperator{\ann}{Ann}
\DeclareMathOperator{\Hom}{Hom}


% some "funny lines" referred to later:
\newtheorem{theorem}{Theorem}[section]
\newtheorem{corollary}[theorem]{Corollary}
\newtheorem{lemma}[theorem]{Lemma}
\newtheorem{proposition}[theorem]{Proposition}
{\theoremstyle{remark}\newtheorem*{remark}{Remark}}
\theoremstyle{definition}
\newtheorem{definition}[theorem]{Definition}
\newtheorem{example}[theorem]{Example}


\newcommand{\lex}{\mbox{lexdeg}}
\newcommand{\mymod}[3]{#1 \equiv #2 \Mod{#3}}
\newcommand{\ccc}{\mathcal{C}}
\newcommand{\nmm}[2]{\text{N}_{#1}(#2)}
\newcommand{\Mod}[1]{\ (\mathrm{mod}\ #1)}
\newcommand{\gal}{\text{Gal}}
\newcommand{\cc}{\mathcal{C}}
\newcommand{\zz}{\mathbb{Z}}
\newcommand{\ta}[1]{\langle #1 \rangle}
\newcommand{\ff}{\mathbb{F}}
\newcommand{\qq}{\mathbb{Q}}
\newcommand{\Tor}[2]{\mathbf{Tor}_{#1}(#2)}
\newcommand{\sqrtn}[1]{\sqrt[n]{#1}}
\newcommand{\charr}{\text{char}}
\newcommand{\disc}[1]{\mbox{disc}(#1)}
\newcommand{\Aut}{\text{Aut}}
\newcommand{\Inn}{\text{Inn}}
\newcommand{\Gal}{\text{Gal}}
\newcommand{\sgn}{\text{sgn}}
\newcommand{\irr}{\text{irr}}
\newcommand{\of}{\overline{F}}
\newcommand{\ok}{\overline{K}}
\newcommand{\ZZ}{\mathbb{Z}}
\newcommand{\NN}{\mathbb{N}}
\newcommand{\CC}{\mathbb{C}}
\newcommand{\QQ}{\mathbb{Q}}
\newcommand{\RR}{\mathbb{R}}
\newcommand{\EE}{\mathbb{E}}
\newcommand{\FF}{\mathbb{F}}
\newcommand{\Tr}{\text{Tr}}
\newcommand{\nm}{\text{N}}
\newcommand{\tk}{\theta_K}
\newcommand{\mm}{\mathfrak{m}}
\newcommand{\tor}{\mathbf{Tor}}
\newcommand{\conv}[1]{\mathrm{conv}(#1)}
\newcommand{\diam}[1]{\mathrm{diam}(#1)}
\newcommand{\vol}[1]{\mathrm{vol}(#1)}
\newcommand{\dist}[2]{\mathrm{dist}(#1, #2)}
\newcommand{\la}{\langle}
\newcommand{\ra}{\rangle}
\newcommand{\inner}[1]{\langle #1 \rangle}

\begin{document}

\title{HW4}

\noindent \textbf{Q1:} Up to rescaling, we can assume the greatest distance is 1. Let $x,y$ be two points realizing distance 1. Then all other points must lies in the bigon, which is the intersection of two unit disk centered at $x,y$. If not, then the point outside the bigon would be at distance $>1$ from $x$ or $y$. Similarly,  if $x,y,z$ are three distinct points with pairwise distance 1, then all other points must lies in the Reuleaux triangle with vertices $x,y,z$.

\begin{center}
  \begin{tikzpicture}[style=thick]
    \draw[fill=yellow!30] (1.732,1) arc[start angle=30, end angle=150,radius=2cm] (-1.732,1) arc[start angle=210, end angle=330,radius=2cm]  (1.732,1)  ;
    \filldraw [black] (0,0) circle (1pt);
    \node at (0,-0.2) {$x$};
    \filldraw [black] (0,2) circle (1pt);
    \node at (0,2.2) {$y$};
    \draw (0,2) -- (0,0);
    % \filldraw [red] (1,1.732) circle (1pt);
    % \draw[dotted] (2,0) -- (1,1.732);
    % \filldraw [red] (15:2) circle (1pt);
    % \filldraw [red] (30:2) circle (1pt);
    % \filldraw [red] (45:2) circle (1pt);
    % \filldraw [red] (50:2) circle (1pt);
    % \filldraw [red] (55:2) circle (1pt);
  \end{tikzpicture}
\end{center}


Let us proceed by induction on $n$. The statement is clearly true for $n=1,2,3$ as $\binom{n}{2}\leq n$ in these cases. Suppose the statement is true for any $n-1$ points on the plane, $n\geq 4$. Let $P_1,\dots, P_n$ be $n$ distinct points on the plane with maximum distance 1, namely,  $\diam{\{P_1,\dots,P_n\}}=1$. Let $G$ be the graph with vertices  $P_1,\dots, P_n$ and  edges  pairs of vertices at distance 1. We want to show $|E(G)|\leq n$. If each vertex has most degree 2, then by the handshaking lemma, there are at most $n$ edges, as required. If there exists some vertex has degree at least 3, then after reordering, we can assume this vertex is $P_1$ and three vertices connected to $P_1$ are $P_2,P_3,P_4$. Note $\diam{\{P_1,\dots,P_n\}}=1$ and $||P_1P_2||=||P_1P_4||=1$, so $\triangle P_1P_2P_4$ is acute. Reordering again if necessary, we can assume $P_3$ lies within the acute angle  $\angle P_2P_1P_4$. Suppose a point $P_i$ is at distance 1 from $P_3$. Since $||P_3P_i||=||P_1P_2||=1$, $P_3P_i$ and $P_1P_2$ must intersect; if not, then some two points from $P_1,P_2,P_3,P_i$ would be at distance $>1$. Similarly, $P_3P_i$ and $P_1P_4$ must intersect. So $P_3P_i$ intersects both $P_1P_2$ and $P_1P_4$. It follows that $P_i=P_1$ is the only possibility. Plainly, $P_1$ is the only point at distance 1 from $P_4$. With $P_4$ removed, we arrive at a subgraph with $n-1$ vertices and one edge less from $G$. By induction hypothesis, the new graph has at most $n-1$ edges. Hence, $|E(G)|\leq n$.


\begin{center}
  \begin{tikzpicture}[style=thick,scale=1.5]
    \draw[fill=yellow!30] (0,0) -- (2,0) arc[start angle=0, end angle=60,radius=2cm] -- (0,0);
    \filldraw  (0,0) circle (1pt);
    \node at (-0.2,0) {$P_1$};
    \filldraw  (2,0) circle (1pt);
    \node at (2.2,0) {$P_2$};
    \filldraw  (1,1.732) circle (1pt);
    \node at (1.2,1.732) {$P_4$};
    \draw[dotted] (2,0) -- (1,1.732);
    %\draw[dotted] (2,0) -- (20:2);
    %\draw[dotted] (1,1.732) -- (20:2);
    \filldraw  (20:2) circle (1pt);
    \node at (20:2.2) {$P_3$};
  \end{tikzpicture}
\end{center}


\newpage
\noindent \textbf{Q2:} It suffices to prove the statement for the standard simplex $\Delta^n\subset \RR^{n+1}$, since every regular simplex can be transformed into the standard simplex by an affine map after changing the ambient space dimension if necessary. Note $\Delta^n$ is the convex hull spanned the canonical basis of $\RR^{n+1}$:$$e_1=(1,0,\dots,0), e_2=(0,1,\dots,0),\dots, e_{n+1} = (0,0,\dots,1).$$ Each $e_i$ is a vertex of $\Delta^n$ and each $\conv{\{e_i,e_j:i\not=j\}}$ is an edge. So there are $\binom{n+1}{2}$ the midpoints of the edges of $\Delta^n$ and they are $\{\frac{1}{2}e_i+\frac{1}{2}e_j:  1\leq i\not=j \leq n+1\}$. For any two distinct midpoints $\frac{1}{2}e_i+\frac{1}{2}e_j, \frac{1}{2}e_{i'}+\frac{1}{2}e_{j'}$, either $i,j,i',j'$ are all distinct or $i=i', j\not=j'$ after reordering. In the former case,  $$\mathrm{dist}( \frac{1}{2}e_i+\frac{1}{2}e_j, \frac{1}{2}e_{i'}+\frac{1}{2}e_{j'} ) = ||\frac{1}{2}e_i+\frac{1}{2}e_j - \frac{1}{2}e_{i'}-\frac{1}{2}e_{j'}||= 1;$$
in the latter case $$\mathrm{dist}( \frac{1}{2}e_i+\frac{1}{2}e_j, \frac{1}{2}e_{i'}+\frac{1}{2}e_{j'} ) = ||\frac{1}{2}e_i+\frac{1}{2}e_j - \frac{1}{2}e_{i'}-\frac{1}{2}e_{j'}|| =||\frac{1}{2}e_j -\frac{1}{2}e_{j'}||  = \frac{\sqrt{2}}{2}.$$
So these midpoints form a 2-distance set.



\newpage
\noindent \textbf{Q3:} Let $R=\sqrt{\frac{5+\sqrt{5}}{10}}$ and $r=\sqrt{\frac{5-\sqrt{5}}{10}}$. Draw the graph on the complex plane and the vertices are $$Re^{\frac{\pi i}{10}},Re^{\frac{5\pi i}{10}},Re^{\frac{9\pi i}{10}},Re^{\frac{13 \pi i}{10}},Re^{\frac{17\pi i}{10}}, re^{\frac{\pi i}{5}}, re^{\frac{3\pi i}{5}},  re^{\pi i},  re^{\frac{7\pi i}{5}}, re^{\frac{9\pi i}{5}}.$$
\begin{center}
  \begin{tikzpicture}[style=thick,scale=0.4]
    \draw (18:8.51cm) -- (90:8.51cm) -- (162:8.51cm) -- (234:8.51cm) -- (306:8.51cm) -- cycle;
    \draw (36:5.26cm) -- (180:5.26cm) -- (324:5.26cm) -- (108:5.26cm) -- (252:5.26cm) -- cycle;
    \foreach \x in {18,90,162,234,306}{
        \draw (\x:8.51cm) -- (\x+90:5.26cm);
        \draw (\x:8.51cm) circle (2pt);
        \draw (\x+90:5.26cm) circle (2pt);
      }
  \end{tikzpicture}
\end{center}


\newpage
\noindent \textbf{Q4:} (a) We prove something stronger:\footnote{Since I know a little bit about the sunflower lemma, it is not of surprise that I know this result as well.}

\begin{theorem}[Oddtown theorem]
  If $\cc$ is a collection of subsets of $[n]$ such that each subset has an odd cardinality and all pairwise intersections have even cardinalities, then $|\cc|\leq n$.
\end{theorem}


In this problem's case, each subset has cardinality 3, which is odd, and the total number of subsets is $n+1>n$. It follows that there must be two sets $A_i$ and $A_j$ with $|A_i\cap A_j|$ odd, hence $|A_i\cap A_j|=1$.

\begin{proof}[Proof of the oddtown theorem]
  For each subset $x\subset [n]$, identify $x$ with a vector $f(x)=(x_1,\dots,x_n)\in \FF_2^n$, where $x_i=1$ if $i\in x$ and $x_i=0$ if $i\notin x$. Indeed, $f$ is a bijection from $2^{[n]}$ to $\FF_2^n$. Now for $x,y\in \FF_2^n$, $\inner{x,y}$ means the number (modulo 2) of coordinates that is 1 in both $x$ and $y$. For each $c\in \ccc$, that $c$ is of odd size is to say $\inner{f(c),f(c)}=1$; for two different $c,d\in \ccc$, that $|c\cap d|$ is even is to say $\inner{f(c),f(d)}=0$. Therefore, $f(\cc)$ is orthonormal and hence  linearly independent. It follows that $|\ccc|=|f(\ccc)|\leq \dim( \FF_2^n ) =n$.
\end{proof}



(b) It is clear the statement is true for $n\leq 4$ as then  $\frac{(n-1)(n-2)}{6}\leq 1$. Assume $n\geq 5$ from now on. Every point $x=(x_1,\dots,x_n)\in V_n$ can be identified with a 3-subset of $[n]$, which is $\{i\in [n]: x_i=1\}$. Denote the identified set by $S(x)$ for each point $x$. (So $S$ is the inverse function of $f$ in part (a).) Note for any two points $x,y\in V_n$, $||x-y||_2=2$ if and only if  $x,y$ differs at exactly 4 coordinates if and only if $|S(x)\cap S(y)|=1$. For each integer $i\in [n]$, there are $\binom{n-1}{4} \binom{4}{2}/2$ such (unordered) pairs of points $x,y$ that $S(x)\cap S(y)=\{i\}$. And so $$|E_n| = 3n\binom{n-1}{4}  \mbox{ and } |V_n|=\binom{n}{3}.$$

By part (a), for every $n+1$ points of $V_n$, say $a_1,\dots,a_{n+1}$, there are $i,j$ such that $|S(a_i)\cap S(a_j)|=1$, that is, $a_i$ and $ a_j$ are connected by an edge. So the independence number $\alpha(G) \leq n$. It follows that $$\chi(G) \geq \frac{|V(G)|}{\alpha(G)} \geq \frac{\binom{n}{3}}{n} = \frac{(n-1)(n-2)}{6}$$ as required.\\


(c) It is clear the statement is true for $n\leq 4$ as then  $\frac{(n-1)(n-2)}{6}\leq 1$. Assume $n\geq 5$. The graph described in part (b) can be draw in $\EE^n$. The problem is that each edge is of length $2$. But this can be resolved by a rescaling $f:\EE^n \to \EE^n,  v\mapsto \frac{1}{2}v$. Now there is an edge in if and only $x$ and $y$ differs at 4 coordinates and hence  $||f(x)-f(y)||=1$. That is $f(G)$ is a unit distance graph in $\EE^n$. Hence, $\chi(\EE^d) \geq \chi(G) \geq   \frac{(n-1)(n-2)}{6}$ as required.

\newpage
\noindent \textbf{Q5:}  Note that $B(p,\epsilon)$ is contained in $reg(p)$ for sufficiently small $\epsilon >0$.


Suppose $p$ lies on the surface of $\conv{P}$. This implies that there is a hyperplane $h$ such that $p\in h$  and $\conv{P}$ lies entirely in one of the half-spaces determined by $h$. Assume the hyperplane is given by $$h=\{x\in \RR^d: \inner{x, n} = b \}$$ for some $n\in \RR^d$ and $b\in \RR$, and changing $n$ to $-n$ if necessary, we can also assume that $ \inner{x, n} \leq b$ for all $x\in \conv{P}$. We claim that $p+tn \in reg(p)$ for all positive number $t$ and hence $reg(p)$ is unbounded. To see the claim, we have the following inequality for all $q\in P$ and $t\geq 0$:
\begin{align*}
  ||q - (p+tn)||^2 - || p-(p+tn) ||^2 & = \inner{q - (p+tn),q - (p+tn)} - \inner{tn,tn}     \\
                                      & = \inner{(q - p) - tn),(q - p)-tn)} - \inner{tn,tn} \\
                                      & =\inner{ q - p,q - p} -2\inner{q - p,tn}            \\
                                      & = ||q-p||^2 +2t\inner{p,n} - 2t\inner{q,n}          \\
                                      & = ||q-p||^2 + 2tb - 2t\inner{q,n}                   \\
                                      & \geq  ||q-p||^2 + 2tb -2tb =||q-p||^2 \geq 0.
\end{align*}
It follows that $||q - (p+tn)|| \geq || p-(p+tn) ||$ for all $q\in P$ and so $p+tn \in reg(p)$ as claimed.\\


For the reverse direction, suppose $reg(p)$ is unbounded. As the intersection of half-spaces, $reg(p)$ is a convex polyhedron. Since $reg(p)$ is convex and unbounded, there exists a ray with initial point $p$ lying entirely in $reg(p)$. Say the ray can be parametrized as $p+tn$ for some $n\in \RR^d$ and parameter $t\in \RR_{\geq 0}$. Set $b= \inner{p,n}$. We claim that the hyperplane $$h=\{x\in \RR^d: \inner{x, n} = b \}$$ intersects with $\conv{P}$  at a face of  $\conv{P}$, or equivalently, $\conv{P}\subset h_+ =\{x\in \RR^d: \inner{x, n} \leq b \}$ and $\conv{P} \cap h_+ \not= \emptyset$. It is clear $\conv{P} \cap h_+ \not= \emptyset$ as $p\in h_+$ and hence $p$ lies ont on surface of  $\conv{P}$ if the claim is true. To prove the claim, we note that if $P\subset h_+$, then $\conv{P}\subset h_+$ as $h_+$ is convex, and so it suffices to show $P\subset h_+$. We suppose by contradiction that there is some $q\in P$ such that $q$ is not in $h_+$, that is, $\inner{q,n}>b$. Note $q\not= p$. Then by exactly the same calculation as above, we have
\begin{align*}
  ||q-(p+tn)||^2  - ||p-(p+tn)||^2 & = ||q-p||^2 + 2tb - 2t\inner{q,n} =  ||q-p||^2 - 2t( \inner{q,n} -b).
\end{align*}
Since $\inner{q,n}>b$, we have $\inner{q,n} -b >\epsilon >0$ for some $\epsilon$. Take some $t> \frac{||q-p||^2}{ 2\epsilon}$ and we have
$$ ||q-(p+tn)||^2  - ||p-(p+tn)||^2  = ||q-p||^2 - 2t( \inner{q,n} -b)  < ||q-p||^2 - 2 \frac{||q-p||^2}{ 2\epsilon}\epsilon=0.$$
But this means $||q-(p+tn)|| < ||p-(p+tn)||$ and so $p+tn\notin reg(p)$, contradicting the fact  the ray $\{b+tn: t\geq 0\}$ lies entirely in $reg(p)$.

% As the intersection of half-spaces containing an $\epsilon$-ball, $reg(p)$ is a convex polyhedron and has at least $d+1$ vertices. Let $V$ be the vertex set of $reg(p)$. By the Carath{\'e}odory's theorem, $p$ can be written as a convex combination of some $v_1,v_2\dots,v_{d+1} \in V$, say $$ p= \sum_{i=1}^{d+1} \lambda_i v_i \mbox{, where each } \lambda_i \geq 0 \mbox{ and } \sum_{i=1}^{d+1}\lambda_i=1.$$ As a vertex in the Voronoi diagram, each vertex is the intersection of $d$ hyperplanes bisecting and orthogonal to the line segments connecting some two points in $P$. Say $v_i$ is the intersection of the hyperplanes bisecting and orthogonal to the line segments joining $p$ to $x_{i,1},x_{i,2},\dots,x_{i,d}$. Then, $\dist{v_i}{p}=\dist{v_i}{x_{i,j}}$ for all $j=1,\dots,d$.




\newpage
\noindent \textbf{Q6:} (a). Denote by $x_1,x_2,x_3$ the centers of these three unit sphere. Then any point $p$ in the intersection of these three spheres is at distance 1 from each $x_i$. In particular, since $p$ is of the same distance from $x_1$ and $x_2$, $p$ lies on the plane bisecting and orthogonal to the line segment $x_1x_2$. Similarly, $p$ lies on the plane bisecting and orthogonal to the line segment $x_1x_3$. The intersection of these two planes is a line and $p$ must lie on this line. But there are at most two points on the line at distance 1 from $x_1$. Hence, the intersection of three unit spheres consists of at most 2 points.\\


(b). Suppose we have $n$ distinct points on $\RR^3$ realizing the maximum number of unit distances. Let $G$ be the graph such that the vertex set of $G$ is these $n$ points and two vertices are connected by an edge if and only if they are at unit distance from each other. Then $G$ contains no subgraph isomorphic to $K_{3,3}$. Otherwise, there were three points in the intersection of three distinct unit spheres, violating part (a). So by the K{\"o}v{\'a}ri-S{\'o}s-Tur{\'a}n theorem, $|E(G)| = O(n^{2-1/3}) = O(n^{5/3})$.

To get the inequality in statement, we need a somewhat more precise estimate, \emph{cf.} Theorem 17.2.5 on the lecture notes. Plainly, $U_3(n) = |E(G)|\leq ex(n,K_{3,3}) \leq (\frac{1}{2}+o(1)) n^{5/3}$.

\end{document}


