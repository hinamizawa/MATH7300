\documentclass[12pt,reqno]{amsart}
\usepackage{fullpage}
\usepackage{amsfonts}
\usepackage{tikz}
\usepackage{amssymb}
\usepackage{times}
\usepackage{graphicx}
\usepackage{mathtools}
\usepackage{breakurl}
\usepackage{bm}
\usepackage{amsmath}
\usepackage{blkarray}
\usepackage{url}
\usepackage[all]{xy}
\usepackage[margin=0.8in,footskip=0.25in]{geometry}

\usepackage[colorlinks=true,
            linkcolor=red,
            urlcolor=blue,
            citecolor=red]{hyperref}
\vfuzz=2pt


\DeclarePairedDelimiter\ceil{\lceil}{\rceil}
\DeclarePairedDelimiter\floor{\lfloor}{\rfloor}

\DeclareMathOperator{\cok}{coker}
\DeclareMathOperator{\im}{im}
\DeclareMathOperator{\ann}{Ann}
\DeclareMathOperator{\Hom}{Hom}


% some "funny lines" referred to later:
\newtheorem{theorem}{Theorem}[section]
\newtheorem{corollary}[theorem]{Corollary}
\newtheorem{lemma}[theorem]{Lemma}
\newtheorem{proposition}[theorem]{Proposition}
{\theoremstyle{remark}\newtheorem*{remark}{Remark}}
\theoremstyle{definition}
\newtheorem{definition}[theorem]{Definition}
\newtheorem{example}[theorem]{Example}


\newcommand{\lex}{\mbox{lexdeg}}
\newcommand{\mymod}[3]{#1 \equiv #2 \Mod{#3}}
\newcommand{\ccc}{\mathcal{C}}
\newcommand{\nmm}[2]{\text{N}_{#1}(#2)}
\newcommand{\Mod}[1]{\ (\mathrm{mod}\ #1)}
\newcommand{\gal}{\text{Gal}}
\newcommand{\cc}{\mathbb{C}}
\newcommand{\zz}{\mathbb{Z}}
\newcommand{\ta}[1]{\langle #1 \rangle}
\newcommand{\ff}{\mathbb{F}}
\newcommand{\qq}{\mathbb{Q}}
\newcommand{\Tor}[2]{\mathbf{Tor}_{#1}(#2)}
\newcommand{\sqrtn}[1]{\sqrt[n]{#1}}
\newcommand{\charr}{\text{char}}
\newcommand{\disc}[1]{\mbox{disc}(#1)}
\newcommand{\Aut}{\text{Aut}}
\newcommand{\Inn}{\text{Inn}}
\newcommand{\Gal}{\text{Gal}}
\newcommand{\sgn}{\text{sgn}}
\newcommand{\irr}{\text{irr}}
\newcommand{\of}{\overline{F}}
\newcommand{\ok}{\overline{K}}
\newcommand{\ZZ}{\mathbb{Z}}
\newcommand{\NN}{\mathbb{N}}
\newcommand{\CC}{\mathbb{C}}
\newcommand{\QQ}{\mathbb{Q}}
\newcommand{\RR}{\mathbb{R}}
\newcommand{\FF}{\mathbb{F}}
\newcommand{\Tr}{\text{Tr}}
\newcommand{\nm}{\text{N}}
\newcommand{\tk}{\theta_K}
\newcommand{\mm}{\mathfrak{m}}
\newcommand{\tor}{\mathbf{Tor}}
\newcommand{\conv}[1]{\mathrm{conv}(#1)}
\newcommand{\diam}[1]{\mathrm{diam}(#1)}
\newcommand{\vol}[1]{\mathrm{vol}(#1)}
\newcommand{\dist}[2]{\mathrm{dist}(#1, #2)}
\newcommand{\la}{\langle}
\newcommand{\ra}{\rangle}

\begin{document}

\title{HW4}

\noindent \textbf{Q1:} Up to rescaling, we can assume the greatest distance is 1. Let $x,y$ be two points realizing distance 1. Then all other points must lies in the bigon, which is the intersection of two unit disk centered at $x,y$. If not, then the point outside the bigon would have distance $>1$ from $x$ or $y$. Similarly,  if $x,y,z$ are three distinct points with pairwise distance 1, then all other points must lies in the Reuleaux triangle with vertices $x,y,z$.

\begin{center}
  \begin{tikzpicture}[style=thick]
    \draw[fill=yellow!30] (1.732,1) arc[start angle=30, end angle=150,radius=2cm] (-1.732,1) arc[start angle=210, end angle=330,radius=2cm]  (1.732,1)  ;
    \filldraw [black] (0,0) circle (1pt);
    \node at (0,-0.2) {$x$};
    \filldraw [black] (0,2) circle (1pt);
    \node at (0,2.2) {$y$};
    \draw (0,2) -- (0,0);
    % \filldraw [red] (1,1.732) circle (1pt);
    % \draw[dotted] (2,0) -- (1,1.732);
    % \filldraw [red] (15:2) circle (1pt);
    % \filldraw [red] (30:2) circle (1pt);
    % \filldraw [red] (45:2) circle (1pt);
    % \filldraw [red] (50:2) circle (1pt);
    % \filldraw [red] (55:2) circle (1pt);
  \end{tikzpicture}
\end{center}


Let us proceed by induction on $n$. The statement is clearly true for $n=1,2,3$ as $\binom{n}{2}\leq 3$ in these cases.


% \begin{center}
%   \begin{tikzpicture}[style=thick,scale=2]
%     \draw[fill=yellow!30] (0,0) -- (2,0) arc[start angle=0, end angle=60,radius=2cm] -- (0,0);
%     \filldraw [red] (0,0) circle (1pt);
%     \filldraw [red] (2,0) circle (1pt);
%     \filldraw [red] (1,1.732) circle (1pt);
%     \draw[dotted] (2,0) -- (1,1.732);
%     \filldraw [red] (15:2) circle (1pt);
%     \filldraw [red] (30:2) circle (1pt);
%     \filldraw [red] (45:2) circle (1pt);
%     \filldraw [red] (50:2) circle (1pt);
%     \filldraw [red] (55:2) circle (1pt);
%   \end{tikzpicture}
% \end{center}


\newpage
\noindent \textbf{Q2:} It suffices to prove the statement for the standard simplex $\Delta^n \RR^{n+1}$, since every regular simplex can be mapped to the standard simplex by an affine map. Note $\Delta^n$ is the convex hull spanned the canonical basis $$e_1=(1,0,\dots,0), e_2=(0,1,\dots,0),\dots, e_{n+1} = (0,0,\dots,1).$$ Each $e_i$ is a vertex of $\Delta^n$ and each $\conv{\{e_i,e_j:i\not=j\}}$ is an edge. So there are $\binom{n+1}{2}$ the midpoints of the edges of $\Delta^n$ and they are of the form $\frac{1}{2}e_i+\frac{1}{2}e_j, i\not=j$. For any two midpoints $\frac{1}{2}e_i+\frac{1}{2}e_j, \frac{1}{2}e_{i'}+\frac{1}{2}e_{j'}$, either $\{i,j\}\not=\{i',j'\}$ or $i=i', j\not=j'$ after reordering. In the former case,  $$\mathrm{dist}( \frac{1}{2}e_i+\frac{1}{2}e_j, \frac{1}{2}e_{i'}+\frac{1}{2}e_{j'} ) = ||\frac{1}{2}e_i+\frac{1}{2}e_j - \frac{1}{2}e_{i'}-\frac{1}{2}e_{j'}||= 1;$$
in the latter case $$\mathrm{dist}( \frac{1}{2}e_i+\frac{1}{2}e_j, \frac{1}{2}e_{i'}+\frac{1}{2}e_{j'} ) = ||\frac{1}{2}e_i+\frac{1}{2}e_j - \frac{1}{2}e_{i'}-\frac{1}{2}e_{j'}|| =||\frac{1}{2}e_j -\frac{1}{2}e_{j'}||  = \frac{\sqrt{2}}{2}.$$
So these midpoints form a 2-distance set.



\newpage
\noindent \textbf{Q3:} Let $R=\sqrt{\frac{5+\sqrt{5}}{10}}$ and $r=\sqrt{\frac{5-\sqrt{5}}{10}}$. Draw the graph on the complex plane and the vertices are $$Re^{\frac{\pi i}{10}},Re^{\frac{5\pi i}{10}},Re^{\frac{9\pi i}{10}},Re^{\frac{13 \pi i}{10}},Re^{\frac{17\pi i}{10}}, re^{\frac{\pi i}{5}}, re^{\frac{3\pi i}{5}},  re^{\pi i},  re^{\frac{7\pi i}{5}}, re^{\frac{9\pi i}{5}}.$$
\begin{center}
  \begin{tikzpicture}[style=thick,scale=0.50]
    \draw (18:8.51cm) -- (90:8.51cm) -- (162:8.51cm) -- (234:8.51cm) -- (306:8.51cm) -- cycle;
    \draw (36:5.26cm) -- (180:5.26cm) -- (324:5.26cm) -- (108:5.26cm) -- (252:5.26cm) -- cycle;
    \foreach \x in {18,90,162,234,306}{
        \draw (\x:8.51cm) -- (\x+90:5.26cm);
        \draw (\x:8.51cm) circle (2pt);
        \draw (\x+90:5.26cm) circle (2pt);
      }
  \end{tikzpicture}
\end{center}


\newpage
\noindent \textbf{Q4:}

\newpage
\noindent \textbf{Q5:}  Note that $B(p,\epsilon)$ is contained in $reg(p)$ for sufficiently small $\epsilon >0$. Suppose $reg(p)\subset \RR^d$ is bounded. As the intersection of half-spaces containing an $\epsilon$-ball, $reg(p)$ is a convex polyhedron and has at least $d+1$ vertices. Let $V$ be the vertex set of $reg(p)$. By the Carath{\'e}odory's theorem, $p$ can be written as a convex combination of some $v_1,v_2\dots,v_{d+1} \in V$, say $$ p= \sum_{i=1}^{d+1} \lambda_i v_i \mbox{, where each } \lambda_i \geq 0 \mbox{ and } \sum_{i=1}^{d+1}\lambda_i=1.$$ As a vertex in the Voronoi diagram, each vertex is the intersection of $d$ hyperplanes bisecting and orthogonal to the line segments connecting some two points in $P$. Say $v_i$ is the intersection of the hyperplanes bisecting and orthogonal to the line segments joining $p$ to $x_{i,1},x_{i,2},\dots,x_{i,d}$. Then, $\dist{v_i}{p}=\dist{v_i}{x_{i,j}}$ for all $j=1,\dots,d$.


Now suppose $reg(p)\subset \RR^d$ is unbounded.


\newpage
\noindent \textbf{Q6:} (a). Denote by $x_1,x_2,x_3$ the centers of these three unit sphere. Then any point $p$ in the intersection of these spheres is at distance 1 from each $x_i$. In particular, since $p$ is of the same distance from $x_1$ and $x_2$, $p$ lies on the plane bisecting and orthogonal to the line segment $x_1x_2$. Similarly, $p$ lies on the plane bisecting and orthogonal to the line segment $x_1x_3$. The intersection of these two planes is a line and $p$ must lie on this line. But there are at most two points on the line is of distance 1 from $x_1$. Hence, the intersection of three unit spheres consists of at most 2 points.\\


(b). Suppose we have $n$ distinct points on $\RR^3$ realizing the maximum number of unit distances. Let $G$ be the graph such that the vertex set of $G$ is these $n$ points and two vertices are connected by an edge if and only if they are of unit distance from each other. Then $G$ contains no subgraph isomorphic to $K_{3,3}$. Otherwise, there were three points in the intersection of three distinct unit spheres, violating part (a). So by the K{\"o}v{\'a}ri-S{\'o}s-Tur{\'a}n theorem, $|E(G)| = O(n^{2-1/3}) = O(n^{5/3})$.

To get the inequality in statement, we need a somewhat more precise estimate, \emph{cf.} Theorem 17.2.5 on the lecture notes. Plainly, $U_3(n) \leq |E(G)|\leq ex(n,K_{3,3}) \leq (\frac{1}{2}+o(1)) n^{5/3}$.

\end{document}


